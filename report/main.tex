%%%%%%%%%%%%%%%%%%%%%%%%%%%%%%%%%%%%%%%%%%%%%%%%%%%%%%%%%%%%%%%%%%%%%%%%%%%%%%%
%%                                                                           %%
%%   Dr Varun Ojha                                                           %%
%%   Lecturer, Department of Computer Science                                %% 
%%   University of Reading, UK                                               %%
%%                                                                           %%
%%%%%%%%%%%%%%%%%%%%%%%%%%%%%%%%%%%%%%%%%%%%%%%%%%%%%%%%%%%%%%%%%%%%%%%%%%%%%%%
%%%%     SETTING STARTS - DO NOT CHANGE Unless your TeX setting require so   %%
%%%%%%%%%%%%%%%%%%%%%%%%%%%%%%%%%%%%%%%%%%%%%%%%%%%%%%%%%%%%%%%%%%%%%%%%%%%%%%%
%%----------------------------------------------------------------------------------
% DO NOT Change this. It is the required setting A4 page, 11pt, onside print, book style
%%----------------------------------------------------------------------------------
\documentclass[a4paper,11pt,oneside]{book}

%%-------------------------------------
%% Page margin settings - % half inch margin all sides (recommended)
%%-------------------------------------
\usepackage[margin=1.2in]{geometry} 

%%-------------------------------------
%% Font settings - % CM San or Ariel (recommended)
%%-------------------------------------
% Switch the following two line off: to revert back to default LaTex font (NOT recommended)
\usepackage{amsfonts}
\renewcommand*\familydefault{\sfdefault}

%%-------------------------------------
%% Math/Definition/Theorem/Algorithm packages settings 
%%-------------------------------------
\usepackage[cmex10]{amsmath}
\usepackage{amssymb}
\usepackage{amsthm}
\newtheorem{mydef}{Definition}
\newtheorem{mytherm}{Theorem}
\usepackage{mdframed}
\usepackage{float}
\usepackage[section]{placeins}

%%-------------------------------------
%% Algorithms/Code Listing environment settings  - 
%% Please do not change these settings
%%-------------------------------------
\usepackage{algorithm}
\usepackage{algpseudocode}
\renewcommand{\algorithmicrequire}{\textbf{Input:}}
\renewcommand{\algorithmicensure}{\textbf{Output:}}
\usepackage[utf8]{inputenc}
\usepackage{listings}
\usepackage{xcolor}
\definecolor{codegreen}{rgb}{0,0.6,0.1}
\definecolor{codegray}{rgb}{0.5,0.5,0.5}
\definecolor{codeblue}{rgb}{0.10,0.00,1.00}
\definecolor{codepurple}{rgb}{0.58,0,0.82}
\definecolor{backcolour}{rgb}{1.0,1.0,1.0}

\lstdefinestyle{java}{
    language=Java,
    basicstyle=\ttfamily\footnotesize,
    keywordstyle=\color{blue}\bfseries,
    commentstyle=\color{gray},
    stringstyle=\color{red},
    numbers=left,
    numberstyle=\tiny,
    stepnumber=1,
    breaklines=true,
    frame=single
}

\lstdefinestyle{gradlestyle}{
    language=bash,
    basicstyle=\ttfamily\footnotesize,
    keywordstyle=\color{blue}\bfseries,   % Parole chiave in blu
    commentstyle=\color{gray},            % Commenti in grigio
    stringstyle=\color{red},              % Stringhe in rosso
    morekeywords={run, tasks},            % Parole chiave extra
    emph={runsmartPlayerMAS,runStrategistMAS,runwaySmarterPlayerMAS}, 
    emphstyle=\color{violet},             % Evidenziazione personalizzata per task Gradle
    numbers=left,
    numberstyle=\tiny\color{gray},
    stepnumber=1,
    breaklines=true,
    frame=single
}

\lstdefinelanguage{Prolog}{
    morekeywords={:-,?,assert, retract, dynamic, consult}, % Parole chiave
    morecomment=[l]\%, % Commenti che iniziano con %
    morestring=[b]', % Stringhe tra apici singoli
    keywordstyle=\color{keywordblue}\bfseries, % Stile parole chiave
    commentstyle=\color{commentgreen}, % Stile commenti
    stringstyle=\color{stringred}, % Stile stringhe
    basicstyle=\ttfamily\small, % Font base
    tabsize=4, % Dimensione del tab
    showstringspaces=false, % Non mostrare spazi nelle stringhe
    frame=single, % Bordo attorno al codice
    breaklines=true, % Abilita il ritorno a capo automatico
}

%%-------------------------------------
%% Graphics/Figures environment settings
%%-------------------------------------
\usepackage{graphicx}
\usepackage{subfigure}
\usepackage{caption}
\usepackage{lipsum}

%%-------------------------------------
%% Table environment settings
%%-------------------------------------
\usepackage{multirow}
\usepackage{rotating}
\usepackage{makecell}
\usepackage{booktabs}
%\usepackage{longtable,booktabs}

%%-------------------------------------
%% List of Abbreviations settings
%%-------------------------------------
\usepackage{enumitem}
\newlist{abbrv}{itemize}{1}
\setlist[abbrv,1]{label=,labelwidth=1in,align=parleft,itemsep=0.1\baselineskip,leftmargin=!}

%%-------------------------------------
%% Bibliography/References settings   - Harvard Style was used in this report
%%-------------------------------------
\usepackage[hidelinks]{hyperref}
\usepackage[comma,authoryear]{natbib}
\renewcommand{\bibname}{References} % DO NOT remove or switch of 

%%-------------------------------------
%% Appendix settings     
%%-------------------------------------
\usepackage[toc]{appendix}
%%%%%%%%%%%%%%%%%%%%%%%%%%%%%%%%%%%%%%%%%%%%%%%%%%%%%%%%%%%%%%%%%%%%%%%%%%%%%%%%%%%%%%%
%%%%                     SETTING ENDS                                            %%%%%%
%%%%%%%%%%%%%%%%%%%%%%%%%%%%%%%%%%%%%%%%%%%%%%%%%%%%%%%%%%%%%%%%%%%%%%%%%%%%%%%%%%%%%%%
\begin{document}

    \captionsetup[figure]{margin=1.5cm,font=small,name={Figure},labelsep=colon}
    \captionsetup[table]{margin=1.5cm,font=small,name={Table},labelsep=colon}
    % \setlipsumdefault{1}
    
    \frontmatter
    
    \begin{titlepage}      
        \begin{center}
            % \includegraphics[width=3cm]{figures/uorlogo.png}\\[0.5cm]
            {\LARGE University of Bologna\\[0.5cm]
            Department of Computer Science}\\[2cm]
			%{\color{blue} \rule{\textwidth}{1pt}}
			
			% -------------------------------
			% You need to edit some details here
			% -------------------------------  
            \linespread{1.2}\huge {
                %%%%%%%%%%%%%%%%%%%%%%%%%%%%
                %TODO: 1 TITLE of Your PROJECT 
                %%%%%%%%%%%%%%%%%%%%%%%%%%%%
                % chnage the following line                
                Report Itelligent System Project            
            }
            \huge {
                %%%%%%%%%%%%%%%%%%%%%%%%%%%%
                %TODO: 1 TITLE of Your PROJECT 
                %%%%%%%%%%%%%%%%%%%%%%%%%%%%
                % chnage the following line                
                BDI 21 player agent 
            }
            \linespread{1}~\\[2cm]
			%{\color{blue} \rule{\textwidth}{1pt}}
            {\Large 
                %%%%%%%%%%%%%%%%%%%%%%%%%%%%
                %TODO: 2 YOUR NAME
                %%%%%%%%%%%%%%%%%%%%%%%%%%%%             
                % chnage the following line
                Matteo Santoro
                % change end             
            }\\[1cm] 
            

            {\large 
                %%%%%%%%%%%%%%%%%%%%%%%%%%%%
                %TODO: 3 YOUR NAME Supervisor's name(s)
                %%%%%%%%%%%%%%%%%%%%%%%%%%%%             
                % change the following line                
                \emph{Supervisor:} Andrea Omicini}\\[1cm] % if applicable
                {\large 
                %%%%%%%%%%%%%%%%%%%%%%%%%%%%
                %TODO: 3 YOUR NAME Supervisor's name(s)
                %%%%%%%%%%%%%%%%%%%%%%%%%%%%             
                % change the following line                
                \emph{Supervisor:} Giavanni Ciatto}\\[1cm] % if applicable
            
    		% PLEASE DO NOT CHANGE THIS TEXT %
            \vfill
            
            \today % Please update this date you can use \date{April 2020} for fixed date
        \end{center}
    \end{titlepage}
    
    
    % -------------------------------------------------------------------
    % Declaration
    % -------------------------------------------------------------------
    \newpage
    \thispagestyle{empty}
    % \chapter*{\Large Declaration}
    % % PLEASE CHANGE THIS TEXT EXCEPT YOUR NAME%
    % % -------------------------------
    % %TODO: PLEASE ONLY UPDATE HERE -- PLEASE WRITE YOUR NAME %    
    % % ------------------------------- 
    % I,
    % %%%%%%%%%%%%%%%%%%%%%%%
    %  Firstname(s) Lastname, % Mandatory part
    % %%%%%%%%%%%%%%%%%%%%%%%
    % of the Department of Computer Science, University of Reading, confirm that this is my own work and figures, tables, equations, code snippets, artworks, and illustrations in this report are original and have not been taken from any other person's work, except where the works of others have been explicitly acknowledged, quoted, and referenced. I understand that if failing to do so will be considered a case of plagiarism. Plagiarism is a form of academic misconduct and will be penalised accordingly. \\
    
    % %% Please delete as appropriate. 
    % \noindent
    % %%%%%%%%%%%%%%%%%%%%%%%%%%%%%%%%%%%%%%%%%%%%%%% 
    % %TODO 1 Consent for example copy -  we will use 
    % I give consent to a copy of my report being shared with future students as an exemplar. \\
    
    % \noindent
    % %%%%%%%%%%%%%%%%%%%%%%%%%%%%%%%%%%%%%%%%%%%%%%% 
    % %TODO 2 Consent to let the report to use use by library for public use
    % I give consent for my work to be made available more widely to members of UoR and public with interest in teaching, learning and research. 
    % %%%%%%%%%%%%%%%%%%%%%%%%%%%%%%%%%%%%%%%%%%%%%%%
    % ~\\[1cm]
    % \begin{flushright}
	% %------------------------------ 
	% % change the following line
    % %TODO: PLEASE UPDATE  Your Name  -------------------------------%
	% Firstname(s) Lastname % Please change it to your name
    
    % \today
    % \end{flushright}

     
    % -------------------------------------------------------------------
    % Abstract and Acknowledgement
    % -------------------------------------------------------------------
    
    %Two resources useful for abstract writing.
% Guidance of how to write an abstract/summary provided by Nature: https://cbs.umn.edu/sites/cbs.umn.edu/files/public/downloads/Annotated_Nature_abstract.pdf %https://writingcenter.gmu.edu/guides/writing-an-abstract
\chapter*{\center \Large  Abstract}
%%%%%%%%%%%%%%%%%%%%%%%%%%%%%%%%%%%%%%
% Replace all text with your text
%%%%%%%%%%%%%%%%%%%%%%%%%%%%%%%%%%%

This is an undergraduate project report template and instruction on how to write a report. It also has some useful examples to use \LaTeX. Do read this template carefully. The number of chapters and their titles may vary depending on the type of project and personal preference. Section titles in this template are illustrative should be updated accordingly. For example, sections named ``A section...'' and ``Example of ...'' should be updated. The number of sections in each chapter may also vary. This template may or may not suit your project. Discuss the structure of your report with your supervisor.

%%%
~\\[1cm]%REMOVE THIS
\noindent\textbf{Guidance on abstract writing:} An abstract is a summary of a report in a single paragraph up to a maximum of 250 words. An abstract should be self-contained, and it should not refer to sections, figures, tables, equations, or references. An abstract typically consists of sentences describing the following four parts: (1) introduction (background and purpose of the project), (2) methods, (3) results and analysis, and (4) conclusions. The distribution of these four parts of the abstract should reflect the relative proportion of these parts in the report itself. An abstract starts with a few sentences describing the project's general field, comprehensive background and context, the main purpose of the project; and the problem statement. A few sentences describe the methods, experiments, and implementation of the project. A few sentences describe the main results achieved and their significance. The final part of the abstract describes the conclusions and the implications of the results to the relevant field.


%%%%%%%%%%%%%%%%%%%%%%%%%%%%%%%%%%%%%%%%%%%%%%%%%%%%%%%%%%%%%%%%%%%%%%%%%s
~\\[1cm]
\noindent % Provide your key words
\textbf{Keywords:} a maximum of five keywords/keyphrase separated by commas

\vfill
\noindent
\textbf{Report's total word count:} we expect a maximum of 20,000 words (excluding reference and appendices) and about 50 - 60 pages. [A good project report can also be written in approximately 10,000 words.]


    % -------------------------------------------------------------------
	% Acknowledgement
	% -------------------------------------------------------------------
   
    % \chapter*{\center \Large  Acknowledgements}
%%%% Update with your text %%%%%%%%%%%%%%%
An acknowledgements section is optional. You may like to acknowledge the support and help of your supervisor(s), friends, or any other person(s), department(s), institute(s), etc. If you have been provided specific facility from department/school acknowledged so.  

   
    
    % -------------------------------------------------------------------
    % Contents, list of figures, list of tables
    % -------------------------------------------------------------------
    
    \tableofcontents
    \listoffigures
    % \listoftables
    % \section{List of Abbreviations}
\chaptermark{List of Abbreviations}
%%%%%%%%%%%%%%%%%%%%%%%%%%%%%%%%%%%
%%  Enter your list of Abbreviation and Symbols in this file
%%%%%%%%%%%%%%%%%%%%%%%%%%%%%%%%%%%
\begin{abbrv}
    
    \item[SMPCS]			School of Mathematical, Physical and Computational Sciences
    
\end{abbrv}
 %  Enter your list of Abbreviation and Symbols in this file
    
    %%%%%%%%%%%%%%%%%%%%%%%%%%%%%%%%%%%%%%%%%%%%%%%%%%%%%%%%%%%%%%%%%%%%%%%%
    %%                                                                    %%  
    %%  Main chapters and sections of your project                        %%  
    %%  Everything from here on needs updates in your own words and works %%
    %%                                                                    %%
    %%%%%%%%%%%%%%%%%%%%%%%%%%%%%%%%%%%%%%%%%%%%%%%%%%%%%%%%%%%%%%%%%%%%%%%%
    \mainmatter
    % Read for preparation of document in LaTex 
    % Lamport, L. (1986), LATEX: A Document Preparation System, Addison-Wesley.
    
    \chapter{Goal/Objectives}
\label{ch:goal} % This how you label a chapter and the key (e.g., ch:into) will be used to refer this chapter ``Introduction'' later in the report. 
% the key ``ch:into'' can be used with command \ref{ch:intor} to refere this Chapter.
\section{Goals of this project:}
\begin{itemize}
    \item Implement a multi-agent system that includes different levels of intelligence for a card-playing bot.
    \item Define and implement autonomous behaviors for the bots, allowing them to make strategic decisions based on the context of the game.
    \item Make an HiLo strategic agent.
    \item Transpose agent decision on a GUI that shows a visual table and cards.
    \item Measure the winning rate of the agents.
\end{itemize}

\section{Usage scenarios}

Smart 21 will be a full simulation of the real game. Users can decide to play 21 for fun or test the ability of the agents by letting them play and analyze their stategies.  

\section{Definition of done}

Final delivery will be a gradle project that has different running options:

\begin{itemize}
    \item 1. Pure 21 game.
    \item 2. Dumb agent that plays with BDI defined by common sense.
    \item 3. HiLo agent that expoilts HiLO algorithm to count cards and decide autonomously how much money bet on a hand.
\end{itemize}

To test the agents implementation will be measured the amount of money won and risked and compared this data in a grafical representation.

\chapter{Background and link to the theory}

(To be written at the end)
• Relevant architectural styles (the ones mentioned in Section 3)
• Relevant interaction patterns (the ones mentioned in Section 3)
• Relevant software frameworks (the ones mentioned in Sections 3 and 4)


\chapter{Design}

\section{Structure (domain entities)}

Which entities need to by modelled to reflect the domain? UML class diagram here with
domain entities and possibly messages being exchanged
\section{Interaction}
How should entities interact with each others? UML activity / sequence diagram and
protocols definitions.
\section{Behaviour}
How should each entity behave? UML State diagram.
\section{Architecture}
How are software pieces organised into software modules? UML component / package /
deployment diagrams, data-flow among components, web API description.


\chapter{Salient implementation details}
Anything potentially interesting / non-trivial and technologies adopted to match the
design. This section is expected to be short in case some documentation (e.g. Javadoc
or Swagger Spec) has been produced for the software artefacts. This this case, the
produced documentation should be referenced here.
\chapter{Validation}
Choose a criterion for the evaluation of the produced software and its compliance
to the requirements above. Description of automated (and manual) tests and their
rationale. In case of a test-driven development, describe tests here and possibly report
the amount of passing tests, the total amount of tests and, possibly, the test coverage.
\chapter{Deployment Instructions}
Explain here how to install and launch the produced software artefacts. Assume the
software must be installed on a totally virgin environment. So, report any configuration
step. Gradle and Docker may be useful here to ensure the deployment and launch
processes to be easy.
\chapter{Usage Examples}
Show how to use the produced software artefacts. Ideally, there should be at least one
example for each scenario proposed above.
\chapter{Conclusions}
Recap what you did.
    % \chapter{Literature Review}
\label{ch:lit_rev} %Label of the chapter lit rev. The key ``ch:lit_rev'' can be used with command \ref{ch:lit_rev} to refer this Chapter.

A literature review chapter can be organized in a few sections with appropriate titles. A literature review chapter might  contain the following:
\begin{enumerate}
    \item A review of the state-of-the-art (include theories and solutions) of the field of research.
    \item A description of the project in the context of existing literature and products/systems.
    \item An analysis of how the review is relevant to the intended application/system/problem.
    \item A critique of existing work compared with the intended work.
\end{enumerate}
Note that your literature review should demonstrate the significance of the project.

% PLEAE CHANGE THE TITLE of this section
\section{Example of in-text citation of references in \LaTeX} 
% Note the use of \cite{} and \citep{}
The references in a report relate your content with the relevant sources, papers, and the works of others. To include references in a report, we \textit{cite} them in the texts. In MS-Word, EndNote, or MS-Word references, or plain text as a list can be used. Similarly, in \LaTeX, you can use the ``thebibliography'' environment, which is similar to the plain text as a list arrangement like the MS word. However, In \LaTeX, the most convenient way is to use the BibTex, which takes the references in a particular format [see references.bib file of this template] and lists them in style [APA, Harvard, etc.] as we want with the help of proper packages.    

These are the examples of how to \textit{cite} external sources, seminal works, and research papers. In \LaTeX, if you use ``\textbf{BibTex}'' you do not have to worry much since the proper use of a bibliographystyle package like ``agsm for the Harvard style'' and little rectification of the content in a BiBText source file [In this template, BibTex are stored in the ``references.bib'' file], we can conveniently generate  a reference style. 

Take a note of the commands \textbackslash cite\{\} and \textbackslash citep\{\}. The command \textbackslash cite\{\} will write like ``Author et al. (2019)'' style for Harvard, APA and Chicago style. The command \textbackslash citep\{\} will write like ``(Author et al., 2019).'' Depending on how you construct a sentence, you need to use them smartly. Check the examples of \textbf{in-text citation} of sources listed here [This template recommends the \textbf{Harvard style} of referencing.]:
\begin{itemize}
    \item \cite{lamport1994latex} has written a comprehensive guide on writing in \LaTeX ~[Example of \textbackslash cite\{\} ].
    \item If \LaTeX~is used efficiently and effectively, it helps in writing a very high-quality project report~\citep{lamport1994latex} ~[Example of \textbackslash citep\{\} ].   
    \item A detailed APA, Harvard, and Chicago referencing style guide are available in~\citep{uor_refernce_style}.
\end{itemize}

\noindent 
Example of a numbered list:
\begin{enumerate}
    \item \cite{lamport1994latex} has written a comprehensive guide on writing in \LaTeX.
    \item If \LaTeX is used efficiently and effectively, it helps in writing a very high-quality project report~\citep{lamport1994latex}.   
\end{enumerate}

% PLEAE CHANGE THE TITLE of this section
\section{Example of ``risk'' of unintentional plagiarism}
Using other sources, ideas, and material always bring with it a risk of unintentional plagiarism. 

\noindent
\textbf{\color{red}MUST}: do read the university guidelines on the definition of plagiarism as well as the guidelines on how to avoid plagiarism~\citep{uor_plagiarism}.




% A possible section of you chapter
\section{Critique of the review} % Use this section title or choose a betterone
Describe your main findings and evaluation of the literature. ~\\

% Pleae use this section
\section{Summary} 
Write a summary of this chapter~\\
% https://guides.library.bloomu.edu/litreview
    % % replace all text with your own text.
% in this template few examples are mention
\chapter{Methodology}
\label{ch:method} % Label for method chapter

We mentioned in Chapter~\ref{ch:into} %[example backward reference to a chapter or section.]
that a project report's structure could follow a particular paradigm. Hence, the organization of a report (effectively the Table of Content of a report) can vary depending on the type of project you are doing. Check which of the given examples suit your project. Alternatively, follow your supervisor's advice.

\section{Examples of the sections of a methodology chapter}
A general report structure is summarised (suggested) in Table~\ref{tab:gen_template}. Table~\ref{tab:gen_template} describes that, in general, a typical report structure has three main parts: (1) front matter, (2) main text, and (3) end matter. %[\textbf{also notice that the preceding sentence is an example of a numbered list in a text body}]. 
The structure of the front matter and end matter will remain the same for all the undergraduate final year project report. However, the main text varies as per the project's needs.
\begin{table}[h!]
    \centering
    \caption{Undergraduate report template structure}
    \label{tab:gen_template}
    \begin{tabular}{llll}     
        \toprule
        \multirow{7}{3cm}{Frontmatter} 
        & & Title Page & \\                  
        & & Abstract &    \\          
        & & Acknowledgements & \\                            
        & & Table of Contents &    \\                                
        & & List of Figures   &    \\                        
        & & List of Tables    &    \\                
        & & List of Abbreviations  &    \\                     
        & &   &    \\                        
        \multirow{7}{3cm}{Main text}
        & Chapter 1 & Introduction   &    \\                         
        & Chapter 2 & Literature Review   &    \\
        & Chapter 3 & Methodology   &    \\
        & Chapter 4 & Results    &    \\
        & Chapter 5 & Discussion and Analysis  &    \\
        & Chapter 6 & Conclusions and Future Work  &    \\        
        & Chapter 7 & Refection  &    \\          
        & &   &    \\                       
        \multirow{2}{3cm}{End matter}
        & & References  &    \\   
        & & Appendices (Optional)  &    \\ 
        & & Index (Optional)  &    \\ 
        \bottomrule
    \end{tabular}
\end{table}

\subsection{Example of a software/Web development main text structure}
\label{subsec:se_chpters}
Notice that the ``methodology'' Chapter of Software/Web development in Table~\ref{tab:soft_eng_temp} takes a standard software engineering paradigm (approach). Alternatively, these suggested sections can be the chapters of their own. Also, notice that ``Chapter 5'' in Table~\ref{tab:soft_eng_temp} is ``Testing and Validation'' which is different from the general report template mentioned in Table~\ref{tab:gen_template}. Check with your supervisor if in doubt.
\begin{table}[h!]
    \centering
    \caption{Example of a software engineering-type report structure}
    \label{tab:soft_eng_temp}
    \begin{tabular}{lll}     
        \toprule                   
        Chapter 1 & Introduction   &    \\        
        Chapter 2 & Literature Review  &    \\                   
        Chapter 3 & Methodology   &    \\
        &               & Requirements specifications   \\
        &               & Analysis   \\
        &               & Design   \\
        &               & Implementations   \\
        Chapter 4 & Testing and Validation  &    \\
        Chapter 5 & Results and Discussion      &    \\
        Chapter 6 & Conclusions and Future Work  &    \\        
        Chapter 7 & Reflection  &    \\                          
        \bottomrule
    \end{tabular}
\end{table}

\subsection{Example of an algorithm analysis main text structure}
Some project might involve the implementation of a state-of-the-art algorithm and its performance analysis and comparison with other algorithms. In that case, the suggestion in Table~\ref{tab:algo_temp} may suit you the best. 
\begin{table}[h!]
    \centering
    \caption{Example of an algorithm analysis type report structure}
    \label{tab:algo_temp}
    \begin{tabular}{lll}     
        \toprule                   
        Chapter 1 & Introduction  &    \\        
        Chapter 2 & Literature Review  &    \\                
        Chapter 3 & Methodology   &    \\
        &               & Algorithms descriptions  \\
        &               & Implementations   \\
        &               & Experiments design   \\
        Chapter 4 & Results       &  \\
        Chapter 5 & Discussion and Analysis  &    \\
        Chapter 6 & Conclusion and Future Work  &    \\        
        Chapter 7 & Reflection  &    \\          
        \bottomrule
    \end{tabular}
\end{table}

\subsection{Example of an application type main text structure}
If you are applying some algorithms/tools/technologies on some problems/datasets/etc., you may use the methodology section prescribed in Table~\ref{tab:app_temp}.  
\begin{table}[h!]
    \centering
    \caption{Example of an application type report structure}
    \label{tab:app_temp}
    \begin{tabular}{lll}     
        \toprule                   
        Chapter 1 & Introduction  &    \\        
        Chapter 2 & Literature Review  &    \\                
        Chapter 3 & Methodology   &    \\
        &               & Problems (tasks) descriptions  \\
        &               & Algorithms/tools/technologies/etc. descriptions  \\        
        &               & Implementations   \\
        &               & Experiments design and setup   \\
        Chapter 4 & Results       &  \\
        Chapter 5 & Discussion and Analysis  &    \\
        Chapter 6 & Conclusion and Future Work  &    \\        
        Chapter 7 & Reflection  &    \\          
        \bottomrule
    \end{tabular}
\end{table}

\subsection{Example of a science lab-type main text structure}
If you are doing a science lab experiment type of project, you may use the  methodology section suggested in Table~\ref{tab:lab_temp}. In this kind of project, you may refer to the ``Methodology'' section as ``Materials and Methods.''
\begin{table}[h!]
    \centering
    \caption{Example of a science lab experiment-type report structure}
    \label{tab:lab_temp}
    \begin{tabular}{lll}     
        \toprule                   
        Chapter 1 & Introduction  &    \\        
        Chapter 2 & Literature Review  &    \\                
        Chapter 3 & Materials and Methods   &    \\
        &               & Problems (tasks) description  \\
        &               & Materials \\        
        &               & Procedures  \\                
        &               & Implementations   \\
        &               & Experiment set-up   \\
        Chapter 4 & Results       &  \\
        Chapter 5 & Discussion and Analysis  &    \\
        Chapter 6 & Conclusion and Future Work  &    \\        
        Chapter 7 & Reflection  &    \\          
        \bottomrule
    \end{tabular}
\end{table}

\section{Example of an Equation in \LaTeX}
Eq.~\ref{eq:eq_example} [note that this is an example of an equation's in-text citation] is an example of an equation in \LaTeX. In Eq.~\eqref{eq:eq_example}, $ s $ is the mean of elements $ x_i \in \mathbf{x} $: 

\begin{equation}
\label{eq:eq_example} % label used to refer the eq in text
s = \frac{1}{N} \sum_{i = 1}^{N} x_i. 
\end{equation}

Have you noticed that all the variables of the equation are defined using the \textbf{in-text} maths command \$.\$, and Eq.~\eqref{eq:eq_example} is treated as a part of the sentence with proper punctuation? Always treat an equation or expression as a part of the sentence. 

\section{Example of a Figure in \LaTeX}
Figure~\ref{fig:chart_a} is an example of a figure in \LaTeX. For more details, check the link:

\href{https://en.wikibooks.org/wiki/LaTeX/Floats,_Figures_and_Captions}{wikibooks.org/wiki/LaTeX/Floats,\_Figures\_and\_Captions}.

\noindent
Keep your artwork (graphics, figures, illustrations) clean and readable. At least 300dpi is a good resolution of a PNG format artwork. However, an SVG format artwork saved as a PDF will produce the best quality graphics. There are numerous tools out there that can produce vector graphics and let you save that as an SVG file and/or as a PDF file. One example of such a tool is the ``Flow algorithm software''. Here is the link for that: \href{http://www.flowgorithm.org/download/}{flowgorithm.org}.
\begin{figure}[ht]
    \centering
    % \includegraphics[scale=0.3]{figures/chart.pdf}
    \caption{Example figure in \LaTeX.}
    \label{fig:chart_a}
\end{figure}

\clearpage %  use command \clearpage when you want section or text to appear in the next page.

\section{Example of an algorithm in \LaTeX}
Algorithm~\ref{algo:algo_example} is a good example of an algorithm in \LaTeX.  
\begin{algorithm}
    \caption{Example caption: sum of all even numbers}
    \label{algo:algo_example}
    \begin{algorithmic}[1]
        \Require{$ \mathbf{x}  = x_1, x_2, \ldots, x_N$}
        \Ensure{$EvenSum$ (Sum of even numbers in $ \mathbf{x} $)}
        \Statex
        \Function{EvenSummation}{$\mathbf{x}$}
        \State {$EvenSum$ $\gets$ {$0$}}
        \State {$N$ $\gets$ {$length(\mathbf{x})$}}
        \For{$i \gets 1$ to $N$}                    
        \If{$ x_i\mod 2 == 0$}  \Comment check if a number is even?
        \State {$EvenSum$ $\gets$ {$EvenSum + x_i$}}
        \EndIf
        \EndFor
        \State \Return {$EvenSum$}
        \EndFunction
    \end{algorithmic}
\end{algorithm}
 
\section{Example of code snippet  in \LaTeX}

Code Listing~\ref{list:python_code_ex} is a good example of including a code snippet in a report. While using code snippets, take care of the following:
\begin{itemize}
    \item do not paste your entire code (implementation) or everything you have coded. Add code snippets only. 
    \item The algorithm shown in Algorithm~\ref{algo:algo_example} is usually preferred over code snippets in a technical/scientific report. 
    \item Make sure the entire code snippet or algorithm stays on a single page and does not overflow to another page(s).  
\end{itemize}

Here are three examples of code snippets for three different languages (Python, Java, and CPP) illustrated in Listings~\ref{list:python_code_ex}, \ref{list:java_code_ex}, and \ref{list:cpp_code_ex} respectively.  

\begin{lstlisting}[language=Python, caption={Code snippet in \LaTeX ~and  this is a Python code example}, label=list:python_code_ex]
import numpy as np

x  = [0, 1, 2, 3, 4, 5] # assign values to an array
evenSum = evenSummation(x) # call a function

def evenSummation(x):
    evenSum = 0
    n = len(x)
    for i in range(n):
        if np.mod(x[i],2) == 0: # check if a number is even?
            evenSum = evenSum + x[i]
    return evenSum
\end{lstlisting}

Here we used  the ``\textbackslash clearpage'' command and forced-out the second listing example onto the next page. 
\clearpage  %
\begin{lstlisting}[language=Java, caption={Code snippet in \LaTeX ~and  this is a Java code example}, label=list:java_code_ex]
public class EvenSum{ 
    public static int evenSummation(int[] x){
        int evenSum = 0;
        int n = x.length;
        for(int i = 0; i < n; i++){
            if(x[i]%2 == 0){ // check if a number is even?
                evenSum = evenSum + x[i];
            }
        }
        return evenSum;     
    }
    public static void main(String[] args){ 
        int[] x  = {0, 1, 2, 3, 4, 5}; // assign values to an array
        int evenSum = evenSummation(x);
        System.out.println(evenSum);
    } 
} 
\end{lstlisting}


\begin{lstlisting}[language=C, caption={Code snippet in \LaTeX ~and  this is a C/C++ code example}, label=list:cpp_code_ex]
int evenSummation(int x[]){
    int evenSum = 0;
    int n = sizeof(x);
    for(int i = 0; i < n; i++){
        if(x[i]%2 == 0){ // check if a number is even?
            evenSum = evenSum + x[i];
    	}
    }
    return evenSum;     
}

int main(){
    int x[]  = {0, 1, 2, 3, 4, 5}; // assign values to an array
    int evenSum = evenSummation(x);
    cout<<evenSum;
    return 0;
}
\end{lstlisting}



\section{Example of in-text citation style}
\subsection{Example of the equations and illustrations placement and reference in the text}
Make sure whenever you refer to the equations, tables, figures, algorithms,  and listings for the first time, they also appear (placed) somewhere on the same page or in the following page(s). Always make sure to refer to the equations, tables and figures used in the report. Do not leave them without an \textbf{in-text citation}. You can refer to equations, tables and figures more them once.

\subsection{Example of the equations and illustrations style}
Write \textbf{Eq.} with an uppercase ``Eq`` for an equation before using an equation number with (\textbackslash eqref\{.\}). Use ``Table'' to refer to a table, ``Figure'' to refer to a figure, ``Algorithm'' to refer to an algorithm and ``Listing'' to refer to listings (code snippets). Note that, we do not use the articles ``a,'' ``an,'' and ``the'' before the words Eq., Figure, Table, and Listing, but you may use an article for referring the words figure, table, etc. in general.

For example, the sentence ``A report structure is shown in \textbf{the} Table~\ref{tab:gen_template}'' should be written as ``A report structure is shown \textbf{in} Table~\ref{tab:gen_template}.'' 
 

\section{Summary}
Write a summary of this chapter.

~\\[5em]
\noindent
{\huge\textbf{Note:}} In the case of \textbf{software engineering} project a Chapter ``\textbf{Testing and Validation}'' should precede the ``Results'' chapter. See Section~\ref{subsec:se_chpters} for report organization of such project. 


    % \chapter{Results}
\label{ch:results}
The results chapter tells a reader about your findings based on the methodology you have used to solve the investigated problem. For example: 
\begin{itemize}
    \item If your project aims to develop a software/web application, the results may be the developed software/system/performance of the system, etc., obtained using a relevant methodological approach in software engineering. 
    
    \item If your project aims to implement an algorithm for its analysis, the results may be the performance of the algorithm obtained using a relevant experiment design. 
    
    \item If your project aims to solve some problems/research questions over a collected dataset, the results may be the findings obtained using the applied tools/algorithms/etc. 
\end{itemize}
Arrange your results and findings in a logical sequence. 



\section{A section}

...

\clearpage
\section{Example of a Table in \LaTeX}
Table~\ref{tab:_ex_tab} is an example of a table created using the package \LaTeX  ``booktabs.'' do check the link: \href{https://en.wikibooks.org/wiki/LaTeX/Tables}{wikibooks.org/wiki/LaTeX/Tables} for more details. A table should be clean and readable. Unnecessary horizontal lines and vertical lines in tables make them unreadable and messy. The example in Table~\ref{tab:_ex_tab} uses a minimum number of liens (only necessary ones). Make sure that the top rule and bottom rule (top and bottom horizontal lines) of a table are present. 

\begin{table}[h!]
    \centering
    \caption{Example of a table in \LaTeX}
    \label{tab:_ex_tab}
    \begin{tabular}{llr}     
        \toprule
        \multicolumn{2}{c}{Bike} \\
        \cmidrule(r){1-2}
        Type    &  Color & Price (\pounds) \\
        \midrule
        Electric    & black   & 700   \\
        Hybrid      & blue    & 500   \\
        Road        & blue    & 300   \\
        Mountain    & red     & 300   \\
        Folding     & black   & 500   \\
        \bottomrule
    \end{tabular}
\end{table}

\section{Example of captions style}

\begin{itemize}
    \item The \textbf{caption of a Figure (artwork) goes below} the artwork (Figure/Graphics/illustration). See example artwork in Figure~\ref{fig:chart_a}. 
    \item  The \textbf{caption of a Table goes above} the table. See the example in Table~\ref{tab:_ex_tab}.
    \item  The \textbf{caption of an Algorithm goes above} the algorithm. See the example in Algorithm~\ref{algo:algo_example}.
    \item The \textbf{caption of a Listing goes below} the Listing  (Code snippet). See example listing in Listing~\ref{list:python_code_ex}. 
\end{itemize} 





\section{Summary}
Write a summary of this chapter.




    % \chapter{Discussion and Analysis}
\label{ch:evaluation}

Depending on the type of project you are doing, this chapter can be merged with ``Results'' Chapter as `` Results and Discussion'' as suggested by your supervisor. 

In the case of software development and the standalone applications, describe the significance of the obtained results/performance of the system. 



\section{A section}% please use an appropriate section title
Discussion and analysis chapter evaluates and analyses the results. It interprets the obtained results. 



\section{Significance of the findings}
In this chapter, you should also try to discuss the significance of the results and key findings, in order to enhance the reader's understanding of the investigated problem

\section{Limitations} % please discuss limitation of the project 
Discuss the key limitations and potential implications or improvements of the findings.
\section{Summary}
Write a summary of this chapter.
    % \chapter{Conclusions and Future Work}
\label{ch:con}
\section{Conclusions}
Typically a conclusions chapter first summarizes the investigated problem and its aims and objectives. It summaries the critical/significant/major findings/results about the aims and objectives that have been obtained by applying the key methods/implementations/experiment set-ups. A conclusions chapter draws a picture/outline of your project's central and the most signification contributions and achievements. 

A good conclusions summary could be approximately 300--500 words long, but this is just a recommendation.

A conclusions chapter followed by an abstract is the last things you write in your project report.

\section{Future work}
This section should refer to Chapter~\ref{ch:results} where the author has reflected their criticality about their own solution. The future work is then sensibly proposed in this section.

\textbf{Guidance on writing future work:} While working on a project, you gain experience and learn the potential of your project and its future works. Discuss the future work of the project in technical terms. This has to be based on what has not been yet achieved in comparison to what you had initially planned and what you have learned from the project. Describe to a reader what future work(s) can be started from the things you have completed. This includes identifying what has not been achieved and what could be achieved. 



A good future work summary could be approximately 300--500 words long, but this is just a recommendation.
    % \chapter{Reflection}
\label{ch:reflection}
%%%%%%%%%%%%%%%%%%%%%%%%%%%%%%%
%% Please remove/replace text below
%%%%%%%%%%%%%%%%%%%%%%%%%%%%%%%
Write a short paragraph on the substantial learning experience. This can include your decision-making approach in problem-solving.

\textbf{Some hints:} You obviously learned how to use different programming languages, write reports in \LaTeX and use other technical tools. In this section, we are more interested in what you thought about the experience. Take some time to think and reflect on your individual project as an experience, rather than just a list of technical skills and knowledge. You may describe things you have learned from the research approach and strategy, the process of identifying and solving a problem, the process research inquiry, and the understanding of the impact of the project on your learning experience and future work.

Also think in terms of:
\begin{itemize}
    \item what knowledge and skills you have developed
    \item what challenges you faced, but was not able to overcome
    \item what you could do this project differently if the same or similar problem would come
    \item rationalize the divisions from your initial planed aims and objectives.
\end{itemize}


A good reflective summary could be approximately 300--500 words long, but this is just a recommendation.

~\\[2em]
\noindent
{\huge \textbf{Note:}} The next chapter is ``\textbf{References},'' which will be automatically generated if you are using BibTeX referencing method. This template uses BibTeX referencing.  Also, note that there is difference between ``References'' and ``Bibliography.'' The list of ``References'' strictly only contain the list of articles, paper, and content you have cited (i.e., refereed) in the report. Whereas Bibliography is a list that contains the list of articles, paper, and content you have cited in the report plus the list of articles, paper, and content you have read in order to gain knowledge from. We recommend to use only the list of ``References.'' 

    

    
    % -------------------------------------------------------------------
    % Bibliography/References  -  Harvard Style was used in this report
    % -------------------------------------------------------------------
    \bibliographystyle{agsm} % Harvard Style 
    
    \bibliography{report/references}  %  Patashnik, O. (1988), BibTEXing. Documentation for general BibTEX users.
    
    % -------------------------------------------------------------------
    % Appendices
    % -------------------------------------------------------------------
    
    % \begin{appendices}
    %     \chapter{An Appendix Chapter (Optional)}
\label{appn:A}
% Optional chapter
Some lengthy tables, codes, raw data, length proofs, etc. which are \textbf{very important but not essential part} of the project report goes into an Appendix. An appendix is something a reader would consult if he/she needs extra information and a more comprehensive understating of the report. Also, note that you should use one appendix for one idea.

An appendix is optional. If you feel you do not need to include an appendix in your report, avoid including it. Sometime including irrelevant and unnecessary materials in the Appendices may unreasonably increase the total number of pages in your report and distract the reader.


    %     \chapter{An Appendix Chapter (Optional)}
\label{appn:B}

...
    % \end{appendices}
    
\end{document}
