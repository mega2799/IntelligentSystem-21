\chapter{Goal/Objectives}
\label{ch:goal} % This how you label a chapter and the key (e.g., ch:into) will be used to refer this chapter ``Introduction'' later in the report. 
% the key ``ch:into'' can be used with command \ref{ch:intor} to refere this Chapter.
\section{Goals of this project:}
\begin{itemize}
    \item Implement a multi-agent system that includes different levels of intelligence for a card-playing bot.
    \item Define and implement autonomous behaviors for the bots, allowing them to make strategic decisions based on the context of the game.
    \item Make an HiLo strategic agent.
    \item Transpose agent decision on a GUI that shows a visual table and cards.
    \item Measure the winning rate of the agents.
\end{itemize}

\section{Usage scenarios}

Smart 21 will be a full simulation of the real game. Users can decide to play 21 for fun or test the ability of the agents by letting them play and analyze their stategies.  

\section{Definition of done}

Final delivery will be a gradle project that has different running options:

\begin{itemize}
    \item 1. Pure 21 game.
    \item 2. Dumb agent that plays with BDI defined by common sense.
    \item 3. HiLo agent that expoilts HiLO algorithm to count cards and decide autonomously how much money bet on a hand.
\end{itemize}

To test the agents implementation will be measured the amount of money won and risked and compared this data in a grafical representation.
